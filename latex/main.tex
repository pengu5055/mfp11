\documentclass[a4paper]{article}
\usepackage[utf8]{inputenc}
\usepackage[slovene]{babel}
\usepackage{graphicx}
\usepackage{hyperref}
\usepackage[nottoc]{tocbibind}
\usepackage{minted}
\usepackage{listings}
\usepackage{caption}
\usepackage{subcaption}
\usepackage{amsmath}
\usepackage{ dsfont }
\usepackage{siunitx}
\usepackage{multimedia}
\usepackage[table,xcdraw]{xcolor}
\setlength\parindent{0pt}

\definecolor{codegreen}{rgb}{0,0.6,0}
\definecolor{codegray}{rgb}{0.5,0.5,0.5}
\definecolor{codepurple}{rgb}{0.58,0,0.82}
\definecolor{backcolour}{rgb}{0.95,0.95,0.92}
\newcommand{\ddd}{\mathrm{d}}
\newcommand\myworries[1]{\textcolor{red}{#1}}
\newcommand{\Dd}[3][{}]{\frac{\ddd^{#1} #2}{\ddd #3^{#1}}}

\lstdefinestyle{mystyle}{
    backgroundcolor=\color{backcolour},   
    commentstyle=\color{codegreen},
    keywordstyle=\color{magenta},
    numberstyle=\tiny\color{codegray},
    stringstyle=\color{codepurple},
    basicstyle=\ttfamily\footnotesize,
    breakatwhitespace=false,         
    breaklines=true,                 
    captionpos=b,                    
    keepspaces=true,                 
    numbers=left,                    
    numbersep=5pt,                  
    showspaces=false,                
    showstringspaces=false,
    showtabs=false,                  
    tabsize=2
}

\lstset{style=mystyle}

\begin{document}
\begin{titlepage}
    \begin{center}
        \includegraphics[]{logo.png}
        \vspace*{3cm}
        
        \Huge
        \textbf{Reševanje PDE z metodo Galerkina}
        
        \vspace{0.5cm}
        \large
        11. naloga pri Matematično-fizikalnem praktikumu

        \vspace{4.5cm}
        
        \textbf{Avtor:} Marko Urbanč (28191096)\ \\
        \textbf{Predavatelj:} prof. dr. Borut Paul Kerševan\ \\
        
        \vspace{2.8cm}
        
        \large
        8.9.2023
    \end{center}
\end{titlepage}
\tableofcontents
\newpage
\section{Uvod}
Če poznamo lastne funkcije diferencialnega operatorja za določeno geometrijo, se reševanje parcialnih
diferencialnih enačb včasih lahko prevede na razvoj po lastnih funkcijah. V tem primeru se lahko
diferencialni operator zapiše kot matrika in enačbo potem rešujemo kot sistem linearnih enačb. Tega lahko
računamo kot vemo in znamo. Zdaj smo to počeli že parkrat.\\

V našem primeru bo fizikalna inspiracija Navier-Stokesova enačba, ki je pravzaprav drugi Newtonov zakon za 
tekočine. Vendar pa je ta enačba zelo zapletena in je še vedno odprt problem, ali sploh obstajajo rešitve
v splošnem. Zato se bomo omejili na preprostejši primer, kjer privzamemo, da imamo enakomeren laminaren tok 
nestisljive tekočine v dolgi ravni cevi pod vplivom stalnega tlačnega gradienta $p'$. V tem primeru se 
Navier-Stokesova enačba poenostavi na Poissonovo enačbo

\begin{equation}
    \nabla^2 \vec{v} = -\frac{p'}{\eta}\>,
    \label{eq:poisson}
\end{equation}

kjer je $\vec{v}$ hitrost tekočine in $\eta$ njena viskoznost. Enačbo rešujemo v notranjosti preseka cevi, medtem ko
je ob stenah hitrost enaka nič. Za pretok velja Poiseuillov zakon

\begin{equation}
    \Phi = \int_S{v\>\ddd S}  = C \frac{p' S^2}{8\pi\eta}\>,
    \label{eq:poiseuille}
\end{equation}

kjer je $S$ presek cevi in $C$ konstanta, ki je odvisna od oblike preseka. Konstanta znaša $C = 1$ za krožni presek.
V našem primeru pa bomo določili konstanto $C$ pa polkrožni presek. Uvedemo nove spremenljivke $\xi = r/R$ in 
$u=v\eta/(p'R^2)$ in nato se problem glasi

\begin{equation}
    \Delta u(\xi,\>\phi) = -1, \qquad u(1,\>\phi) = u(\xi,\>0) = u(\xi,\>\pi) = 0\>,
    \label{eq:poisson2}
\end{equation}
\begin{equation}
    C = 8\pi \iint{\frac{u(\xi,\>\phi)\xi\>\ddd\xi\> \ddd\phi}{(\pi/2)^2}}\>.
\end{equation}

Da se izognemo 


\section{Naloga}

\section{Opis reševanja}

\section{Rezultati}


\section{Komentarji in izboljšave}


\newpage
\bibliographystyle{unsrt}
\bibliography{sources}
\end{document}
