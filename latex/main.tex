\documentclass[a4paper]{article}
\usepackage[utf8]{inputenc}
\usepackage[slovene]{babel}
\usepackage{graphicx}
\usepackage{hyperref}
\usepackage[nottoc]{tocbibind}
\usepackage{minted}
\usepackage{listings}
\usepackage{caption}
\usepackage{subcaption}
\usepackage{amsmath}
\usepackage{ dsfont }
\usepackage{siunitx}
\usepackage{multimedia}
\usepackage[table,xcdraw]{xcolor}
\setlength\parindent{0pt}

\definecolor{codegreen}{rgb}{0,0.6,0}
\definecolor{codegray}{rgb}{0.5,0.5,0.5}
\definecolor{codepurple}{rgb}{0.58,0,0.82}
\definecolor{backcolour}{rgb}{0.95,0.95,0.92}
\newcommand{\ddd}{\mathrm{d}}
\newcommand\myworries[1]{\textcolor{red}{#1}}
\newcommand{\Dd}[3][{}]{\frac{\ddd^{#1} #2}{\ddd #3^{#1}}}

\lstdefinestyle{mystyle}{
    backgroundcolor=\color{backcolour},   
    commentstyle=\color{codegreen},
    keywordstyle=\color{magenta},
    numberstyle=\tiny\color{codegray},
    stringstyle=\color{codepurple},
    basicstyle=\ttfamily\footnotesize,
    breakatwhitespace=false,         
    breaklines=true,                 
    captionpos=b,                    
    keepspaces=true,                 
    numbers=left,                    
    numbersep=5pt,                  
    showspaces=false,                
    showstringspaces=false,
    showtabs=false,                  
    tabsize=2
}

\lstset{style=mystyle}

\begin{document}
\begin{titlepage}
    \begin{center}
        \includegraphics[]{logo.png}
        \vspace*{3cm}
        
        \Huge
        \textbf{Reševanje PDE z metodo Galerkina}
        
        \vspace{0.5cm}
        \large
        11. naloga pri Matematično-fizikalnem praktikumu

        \vspace{4.5cm}
        
        \textbf{Avtor:} Marko Urbanč (28191096)\ \\
        \textbf{Predavatelj:} prof. dr. Borut Paul Kerševan\ \\
        
        \vspace{2.8cm}
        
        \large
        8.9.2023
    \end{center}
\end{titlepage}
\tableofcontents
\newpage
\section{Uvod}
Če poznamo lastne funkcije diferencialnega operatorja za določeno geometrijo, se reševanje parcialnih
diferencialnih enačb včasih lahko prevede na razvoj po lastnih funkcijah. V tem primeru se lahko
diferencialni operator zapiše kot matrika in enačbo potem rešujemo kot sistem linearnih enačb. Tega lahko
računamo kot vemo in znamo. Zdaj smo to počeli že parkrat.\\

V našem primeru bo fizikalna inspiracija Navier-Stokesova enačba, ki je pravzaprav drugi Newtonov zakon za 
tekočine. Vendar pa je ta enačba zelo zapletena in je še vedno odprt problem, ali sploh obstajajo rešitve
v splošnem. Zato se bomo omejili na preprostejši primer, kjer privzamemo, da imamo enakomeren laminaren tok 
nestisljive tekočine v dolgi ravni cevi pod vplivom stalnega tlačnega gradienta $p'$. V tem primeru se 
Navier-Stokesova enačba poenostavi na Poissonovo enačbo

\begin{equation}
    \nabla^2 \vec{v} = -\frac{p'}{\eta}\>,
    \label{eq:poisson}
\end{equation}

kjer je $\vec{v}$ hitrost tekočine in $\eta$ njena viskoznost. Enačbo rešujemo v notranjosti preseka cevi, medtem ko
je ob stenah hitrost enaka nič. Za pretok velja Poiseuillov zakon

\begin{equation}
    \Phi = \int_S{v\>\ddd S}  = C \frac{p' S^2}{8\pi\eta}\>,
    \label{eq:poiseuille}
\end{equation}

kjer je $S$ presek cevi in $C$ konstanta, ki je odvisna od oblike preseka. Konstanta znaša $C = 1$ za krožni presek.
V našem primeru pa bomo določili konstanto $C$ pa polkrožni presek. Uvedemo nove spremenljivke $\xi = r/R$ in 
$u=v\eta/(p'R^2)$ in nato se problem glasi

\begin{equation}
    \Delta u(\xi,\>\phi) = -1, \qquad u(1,\>\phi) = u(\xi,\>0) = u(\xi,\>\pi) = 0\>,
    \label{eq:poisson2}
\end{equation}
\begin{equation}
    C = 8\pi \iint{\frac{u(\xi,\>\phi)\xi\>\ddd\xi\> \ddd\phi}{(\pi/2)^2}}\>.
\end{equation}

Da se izognemo računanju lastnih funkcij (v temu primeru Besselovih) in njihovih ničel, lahko zapišemo
rešitev v obliki razvoja po neki poskuni bazi. V našem primeru bomo vzeli bazo

\begin{equation}
    \psi_{nm}(\xi,\>\phi) = \xi^{2m+1}(1-\xi)^n \sin((2m+1)\phi)\>.
\end{equation}

Z njo lahko zapišemo aproksimacijo rešitve kot linearno kombinacijo 

\begin{equation}
    \tilde{u}(\xi,\>\phi) = \sum_{n=1}^{\infty}\sum_{m=0}^{\infty}{a_{nm}\psi_{nm}(\xi,\>\phi)}\>.
\end{equation}

Za te funkcije niti ni nujno da so prava baza v smislu, da bi bile ortogonalne druga na drugo. Potrebno je
da zadoščajo robnim pogojem tako, da jim bo avtomatično zadoščala tudi linearna kombinacija. Približna funkcija 
$\tilde{u}$ seveda ne zadosti Poissonovi enačbi za res, ampak ji preostane majhna napaka $\varepsilon$

\begin{equation}
    \Delta \tilde{u}(\xi,\>\phi) = -1 + \varepsilon(\xi,\>\phi)\>.
\end{equation}

Pri metodi Galerkina zahtevamo, da je napaka ortogonalna na vse poskusne funkcije $\psi_{nm}$, torej

\begin{equation}
    \langle \psi_{nm},\>\varepsilon \rangle = 0 \qquad \forall n,m\>.
\end{equation}

V splošnem bi lahko zahtevali še ortogonalnost napake na nek drug sistem utežnih funkcij. Metoda Galerkina je
poseben primer takih metod (angl. \textit{Methods of Weighted Residuals}), kjer je utežna funkcija kar poskusna
funkcija sama. Omenjena izbira vodi do sistema enačb za koeficiente $a_{nm}$

\begin{equation}
    {A_{nm,n'm'} a_{n'm'}} = b_{nm}\>.
\end{equation}

Koeficiente $b_{nm}$ dobimo iz skalarnega produkta

\begin{equation}
    b_{nm} = \langle -1,\>\psi_{nm} \rangle\>,
\end{equation}

ki se zaradi ortogonalnosti poskusnih funkcij poenostavi v

\begin{equation}
    b_{nm} = - \frac{2}{2m + 1} \mathrm{B}(2m+3, n+1)\>,
\end{equation}

kjer je $\mathrm{B}$ Eulerjeva beta funkcija. Matrika $A$ pa je definirana kot 

\begin{equation}
    A_{nm,n'm'} = \langle \nabla^2\psi_{nm},\>\psi_{n'm'} \rangle\>,
\end{equation}

kar se po upoštevanju ortogonalnosti poskusnih funkcij poenostavi v

\begin{equation}
    A_{nm,n'm'} = -\frac{\pi}{2} \frac{nn'(3+4m)}{2+4m+n+n'} \mathrm{B}(n+n'-1, 3+4m)\>\delta_{mm'}\>.
\end{equation}

Končno, se naša enačba za koeficient za pretok $C$ glasi

\begin{equation}
    C = -\frac{32}{\pi}\sum_{mn,m'n'}{b_{nm}A^{-1}_{nm,n'm'}b_{m'n'}}\>.
    \label{eq:C}
\end{equation}

\section{Naloga}
Naloga od nas zahteva, da rešimo Poissonovo enačbo (\ref{eq:poisson2}) in izračunamo koeficient za 
pretok $C$ v primeru polkrožnega preseka cevi. Naj si tudi pogledamo kako je odvisna natančnost
rešitve od števila členov v indeksih $m$ in $n$.\\
\section{Opis reševanja}
Za reševanje sem prižgal svoj trusty IBM PC/XT 5160 (\textit{beri:} v resnici sem uporabil svoj domači računalnik) in 
preko božjega čudeža uspel uporabiti Python 3.11.0 kljub temu, da je bil ta napisan šele leta 2023. Uporabil sem 
standardni nabor paketov za znanstveno računanje, torej \texttt{numpy}, \texttt{scipy} in \texttt{matplotlib}.\\

Naloge sem se lotil kot zadnjih dveh nalog, kjer smo tudi reševali parcialne diferencialne enačbe, le na druge 
načine. Ne vem če je res smiselno ponavljati iste razloge zakaj uberem takšno metodo reševanja. Več naj bi bilo eventually
dostopno na spletu na \href{https://pengu5055.github.io/}{moji strani}. Napisal sem razred \texttt{GalerkinObject} 
(da ni stalno Solver) v katerem je vsebovano vse kar potrebujemo za reševanje. Ob kreaciji novega \texttt{GalerkinObjekt}-a 
uporabnik poda, kolikšno naj bo število členov v indeksih $m$ in $n$. Ostalo pa se ob klicanju metode \texttt{solve()} 
izračuna samodejno. Imel sem izbiro, da bi lahko uporabil za matrične sisteme kar \texttt{numpy}-jeve funkcije, 
ampak se mi je zdelo dovolj preprosto, da sem kar sam napisal funkcije za reševanje sistema.

\section{Rezultati}
Glede na to, kako se mi, zaradi lastnih napak, mudi z oddajo se mi zdi najbolj smiselno, da gremo kar takoj na rezultate.

\subsection{Bazne funkcije}
Najprej si poglejmo kako izgledajo bazne funkcije. Kar se tiče $\xi$ in $\phi$ mreže sem uporabil v obeh primerih
$1000$ točk v vsaki smeri. V $\xi$ smeri sem vzel interval $[0,\>1]$ v $\phi$ smeri pa $[0,\>\pi]$. Spodaj je prikazanih nekaj 
baznih funkcij za različne $m$ in $n$. Opazimo, da večanje $m$ povečuje število nekih tokovih kanalov porazdeljenih po 
$\phi$ smeri. Večanje $n$ pa centrirano funkcijo potiska vedno bolj proti spodnjemu ravnemu robu cevi. Definitely sem
narisal te grafe na IBM PC/XT 5160.. yes.

% Subplot of two figures
\begin{figure}[H]
    \centering
    \makebox[\textwidth][c]{%
    \begin{minipage}{0.90\textwidth}
        \centering
        \includegraphics[width=\linewidth]{../images/proc/basis_0-1.jpg}
        \caption{$\psi_{01}$}
        \label{fig:psi_0_1}
    \end{minipage}%
    %\hspace{0.001\textwidth}% 
    \begin{minipage}{0.90\textwidth}
        \centering
        \includegraphics[width=\linewidth]{../images/proc/basis_0-2.jpg}
        \caption{$\psi_{02}$}
        \label{fig:psi_0_2}
    \end{minipage}%
    }
    \caption{Bazne funkcije za $m=0$ in $n=1$ ter $n=2$.}
    \label{fig:psi_0}
\end{figure}
\begin{figure}[H]
    \centering
    \makebox[\textwidth][c]{%
    \begin{minipage}{0.90\textwidth}
        \centering
        \includegraphics[width=\linewidth]{../images/proc/basis_1-1.jpg}
        \caption{$\psi_{11}$}
        \label{fig:psi_1_1}
    \end{minipage}%
    %\hspace{0.001\textwidth}% 
    \begin{minipage}{0.90\textwidth}
        \centering
        \includegraphics[width=\linewidth]{../images/proc/basis_1-2.jpg}
        \caption{$\psi_{12}$}
        \label{fig:psi_1_2}
    \end{minipage}%
    }
    \caption{Bazne funkcije za $m=1$ in $n=1$ ter $n=2$.}
    \label{fig:psi_1}
\end{figure}
\begin{figure}[H]
    \centering
    \makebox[\textwidth][c]{%
    \begin{minipage}{0.90\textwidth}
        \centering
        \includegraphics[width=\linewidth]{../images/proc/basis_0-3.jpg}
        \caption{$\psi_{03}$}
        \label{fig:psi_0_3}
    \end{minipage}%
    %\hspace{0.001\textwidth}% 
    \begin{minipage}{0.90\textwidth}
        \centering
        \includegraphics[width=\linewidth]{../images/proc/basis_7-10.jpg}
        \caption{$\psi_{710}$}
        \label{fig:psi_7_10}
    \end{minipage}%
    }
    \caption{Bazne funkcije za $m=0$, $n=3$ in $m=7$ ter $n=10$.}
    \label{fig:psi_2}
\end{figure}
\newpage
\subsection{Koeficient $C$ in pretok}
Sedaj si poglejmo kako je odvisen koeficient $C$ od števila členov v indeksih $m$ in $n$. Kar se tiče $\xi$ in $\phi$ mreže
sem uporabil takšno kot prej. Za izračun koeficienta $C$ sem uporabil enačbo (\ref{eq:C}). 
\begin{figure}[H]
    \centering
    \makebox[\textwidth][c]{%
    \begin{minipage}{0.90\textwidth}
        \centering
        \includegraphics[width=\linewidth]{../images/proc/m_scaling.jpg}
    \end{minipage}%
    %\hspace{0.001\textwidth}% 
    \begin{minipage}{0.90\textwidth}
        \centering
        \includegraphics[width=\linewidth]{../images/proc/m_scaling_time.jpg}
    \end{minipage}%
    }
    \caption{Skaliranje koeficienta $C$ in časa računanja z $m$.}
\end{figure}

Očitno neka srednja vrednost $m$ zelo pripomore k izboljšavi rezultata. Na grafu je narisan primer za $\psi_{1010}$ kot primerjalna vrednost
za $C$. Čas računanja je očitno linearno odvisen od $m$, $n$ pa določa strmino premice. Vrednosti se asimptotsko približujejo neki vrednosti
z večanjem $m$. Sklepam, da je to najbrž prava vrednost $C$.

\begin{figure}[H]
    \centering
    \makebox[\textwidth][c]{%
    \begin{minipage}{0.90\textwidth}
        \centering
        \includegraphics[width=\linewidth]{../images/proc/n_scaling.jpg}
    \end{minipage}%
    %\hspace{0.001\textwidth}% 
    \begin{minipage}{0.90\textwidth}
        \centering
        \includegraphics[width=\linewidth]{../images/proc/n_scaling_time.jpg}
    \end{minipage}%
    }
    \caption{Skaliranje koeficienta $C$ in časa računanja z $n$.}
\end{figure}

Zanimivo je, da $n$ precej drastično vpliva na vrednost $C$. Določi kot stopnica, kje bo vrednost $C$ asimptotsko konvergirala, ampak 
po nekaj členih ne kaže, da bi se vrednost kaj veliko spreminjala, kar pomeni, da pridemo hitro do rezultata. To vse je precej smiselno, 
še sploh bo zdaj, ko si pogledamo tokovna polja, ki so zelo podobi (a ne enaki) kot bazne funkcije za $m=0$ in različne $n$. Verjetno 
te prispevajo glavnino k pretoku. Poglejmo še tokovna polja

\begin{figure}[H]
    \centering
    \makebox[\textwidth][c]{%
    \begin{minipage}{0.90\textwidth}
        \centering
        \includegraphics[width=\linewidth]{../images/proc/ff_1-1.jpg}
    \end{minipage}%
    %\hspace{0.001\textwidth}% 
    \begin{minipage}{0.90\textwidth}
        \centering
        \includegraphics[width=\linewidth]{../images/proc/ff_2-2.jpg}
    \end{minipage}%
    }
    \caption{Tokova polja za $1\times1$ in $2\times2$ bazo.}
\end{figure}

\begin{figure}[H]
    \centering
    \makebox[\textwidth][c]{%
    \begin{minipage}{0.90\textwidth}
        \centering
        \includegraphics[width=\linewidth]{../images/proc/ff_3-3.jpg}
    \end{minipage}%
    %\hspace{0.001\textwidth}% 
    \begin{minipage}{0.90\textwidth}
        \centering
        \includegraphics[width=\linewidth]{../images/proc/ff_10-10.jpg}
    \end{minipage}%
    }
    \caption{Tokova polja za $3\times3$ in $10\times10$ bazo.}
\end{figure}

Kot pričakovano se z večanjem baze rezultat izbolšuje. Pri $10\times10$ bazi imamo že res lepo prikazano tokovno polje. Za praktične 
namene pa je zdaj pravzaprav odvisno od uporabnika in njegove željene natančnosti. Za natančnost na $2$ decimalni mesti je dovolj
že $3\times3$ baza. Za kaj več bi bilo pa treba ugotavljati naprej. Največja vrednost baze za katero sem jaz računal je bila $10\times10$,
to je seveda ker sem na IBM PC/XT 5160 računal in definitivno ne zaradi tega, ker bi se mi mudilo z oddajo. Dobil sem končno vrednost nekje

\begin{equation}
    C = 0.7576\>.
\end{equation}

\section{Komentarji in izboljšave}
Naloga je bila precej zabavna kljub vsemu hitenju zaradi roka oddaje. Med računanjem sem se spomnil navdiha za theme plot-ov in upam,
da je vam bralcu všeč. Sem poskusil uravnotežiti med tem, da zgleda avtentično, lepo in berljivo in da ne porabim ogromno časa, ker ga nimam.
Definitivno bi poskusil več primerjav narediti, še sploh za večje baze. Mogoče bi bilo smiselno pogledati kakšne residume, kaj pa vem. No ja,
upam, da je bilo branje dokaj prijetno in da so bile slike a trip down memory lane in ne kot annoyance.\\
\end{document}
